%!TEX root = ../template.tex
%%%%%%%%%%%%%%%%%%%%%%%%%%%%%%%%%%%%%%%%%%%%%%%%%%%%%%%%%%%%%%%%%%%
%% chapter1.tex
%% NOVA thesis document file
%%
%% Chapter with introduction
%%%%%%%%%%%%%%%%%%%%%%%%%%%%%%%%%%%%%%%%%%%%%%%%%%%%%%%%%%%%%%%%%%%

\typeout{NT FILE chapter1.tex}%

\chapter{Introduction}
\label{cha:introduction}

\prependtographicspath{{Chapters/Figures/Covers/}}

% epigraph configuration
\epigraphfontsize{\small\itshape}
\setlength\epigraphwidth{12.5cm}
\setlength\epigraphrule{0pt}

\includegraphics[width=0.1\linewidth]{NOVAthesisFiles/Images/novathesis-insignia}\hfill
\includegraphics[width=0.875\linewidth]{NOVAthesisFiles/Images/novathesis-text}

\noindent This is the \gls{novathesis} \LaTeX\ template \ntindex[Template!]{Version} \novathesisversion\ from   {Template!date}\novathesisdate.

\epigraph{
  This work is licensed under the \href{https://www.latex-project.org/lppl/lppl-1-3c/}{\LaTeX\ Project Public License v1.3c}.
  To view a copy of this \ntindex[Template!]{license}, visit the \href{https://www.latex-project.org/lppl/}{LaTeX project public license}.
}

\section{Motivation}
\label{sec:Motivation}

The ever-growing demand for efficient and accurate DNA sequencing solutions has highlighted the limitations of traditional methods, particularly in high-throughput environments. Sanger sequencing remains a widely used technique due to its reliability and precision. However, the manual verification of alignment results introduces bottlenecks that hinder the scalability and efficiency of laboratories, especially those handling extensive genomic datasets.

This bottleneck is not merely an operational issue but a scientific challenge. As genomic research expands its applications in fields like medicine, agriculture, and biotechnology, the need for automated tools capable of minimizing errors and optimizing workflows becomes increasingly evident. By integrating artificial intelligence (AI) into these processes, laboratories can achieve new levels of efficiency and accuracy, ultimately advancing the frontiers of genomic research.


\section{Context}
\label{sec:Context}

\ntindex[Context]{}

Sanger sequencing, developed in the 1970s, revolutionized genetic research and remains a gold standard for DNA analysis. The ABI3730xl sequencer is one of the most prominent instruments used for this purpose, providing high-quality sequence alignments. However, the process is not immune to errors, particularly in labeling and detecting sequence anomalies within 96-well plates.

In recent years, artificial intelligence has emerged as a powerful tool for tackling complex problems across various domains, including genomics. Machine learning models, trained on extensive datasets, can identify patterns and anomalies with a precision that surpasses human capabilities. Applying these advancements to the verification phase of Sanger sequencing presents an opportunity to overcome existing limitations and enhance the reliability of DNA sequencing results.


\section{Objectives}
\label{sec:Objectives}

The primary goal of this thesis is to develop a tool that automates the verification of alignment results obtained from Sanger sequencing. The specific objectives include:

\begin{itemize}
  \item Designing an AI algorithm capable of detecting anomalies and homologous sequences in alignment data.
  \item Training the algorithm using real-world datasets from ABI3730xl sequencers 96-well plates.
  \item Designing an AI algorithm capable of detecting anomalies and homologous sequences in alignment data.
  \item Training the algorithm using real-world datasets from ABI3730xl sequencers 96-well plates.
  \item Validating the tools performance against traditional manual verification methods.
  \item Deploying the tool via an online platform to ensure accessibility for laboratories worldwide.
\end{itemize}

\section{Structure}
\label{sec:Structure}

This document is organized as follows:

\begin{itemize}
  \item Introduction: Provides the motivation, context, objectives, and structure of the thesis.
  \item State of the Art: 
  \item Related Work: Reviews existing literature and technologies in Sanger sequencing and AI applications in genomics.
  \item Methodology: Details the design and implementation of the proposed AI driven tool, including data collection, model training, and deployment.
  \item Planning: Presents the planning of the proposed work, composed by desired checkpoints and finish dates.
  \end{itemize}
% subsection suggestions_bugs_and_feature_requests (end)
