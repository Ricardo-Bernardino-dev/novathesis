%!TEX root = ../template.tex
%%%%%%%%%%%%%%%%%%%%%%%%%%%%%%%%%%%%%%%%%%%%%%%%%%%%%%%%%%%%%%%%%%%
%% chapter1.tex
%% NOVA thesis document file
%%
%% Chapter with introduction
%%%%%%%%%%%%%%%%%%%%%%%%%%%%%%%%%%%%%%%%%%%%%%%%%%%%%%%%%%%%%%%%%%%

\typeout{NT FILE chapter1.tex}%

\chapter{Introduction}
\label{cha:introduction}

\prependtographicspath{{Chapters/Figures/Covers/}}

% epigraph configuration
\epigraphfontsize{\small\itshape}
\setlength\epigraphwidth{12.5cm}
\setlength\epigraphrule{0pt}

\includegraphics[width=0.1\linewidth]{NOVAthesisFiles/Images/novathesis-insignia}\hfill
\includegraphics[width=0.875\linewidth]{NOVAthesisFiles/Images/novathesis-text}

\noindent This is the \gls{novathesis} \LaTeX\ template \ntindex[Template!]{Version} \novathesisversion\ from   {Template!date}\novathesisdate.

\epigraph{
  This work is licensed under the \href{https://www.latex-project.org/lppl/lppl-1-3c/}{\LaTeX\ Project Public License v1.3c}.
  To view a copy of this \ntindex[Template!]{license}, visit the \href{https://www.latex-project.org/lppl/}{LaTeX project public license}.
}

\section{Motivation}
\label{sec:Motivation}

The ever-growing demand for efficient and accurate DNA sequencing solutions has highlighted the limitations of traditional methods, particularly in high-throughput environments. Sanger sequencing remains a widely used technique due to its reliability and precision. However, the manual verification of alignment results introduces bottlenecks that hinder the scalability and efficiency of laboratories, especially those handling extensive genomic datasets.

This bottleneck is not merely an operational issue but a scientific challenge. As genomic research expands its applications in fields like medicine, agriculture, and biotechnology, the need for automated tools capable of minimizing errors and optimizing workflows becomes increasingly evident. By integrating artificial intelligence (AI) into these processes, laboratories can achieve new levels of efficiency and accuracy, ultimately advancing the frontiers of genomic research.

\section{Context}
\label{sec:Context}

\ntindex[Context]{}

Deoxyribonucleic acid (DNA) is the fundamental molecule that encodes the genetic information of all living organisms. It is composed of two complementary strands that form a double helix, with each strand consisting of sequences of four nucleotides: adenine (A), thymine (T), cytosine (C), and guanine (G). The order of these nucleotides determines the genetic instructions necessary for the development, function, and reproduction of organisms.

DNA sequencing is the process of determining the precise order of nucleotides within a DNA molecule. This technology has revolutionized biology and medicine, enabling advancements in areas such as genetics, evolutionary studies, and personalized medicine. Among the various sequencing methods, Sanger sequencing remains one of the most widely used techniques due to its accuracy and reliability.

Sanger sequencing, developed by Frederick Sanger in the 1970s, utilizes the selective incorporation of chain-terminating dideoxynucleotides during DNA replication. This method produces a collection of DNA fragments of varying lengths, which are then separated by capillary electrophoresis. The sequence is determined by detecting the fluorescently labeled nucleotides at the ends of these fragments. Despite the emergence of high-throughput sequencing technologies, Sanger sequencing is still extensively used for smaller-scale projects and applications requiring high precision, such as mutation analysis and microbial identification.

The ABI3730xl is a high-capacity DNA analyzer widely employed in laboratories for Sanger sequencing. This instrument can process up to 96 samples simultaneously, making it a cornerstone for high-throughput DNA sequencing. The sequencer automates many steps of the process, including electrophoresis and data collection. However, the subsequent analysis and verification of sequence alignments often require significant manual effort. Errors such as mislabeling and sequence anomalies within the 96-well plates can arise, necessitating careful scrutiny to ensure data accuracy.

Artificial intelligence (AI) and machine learning (ML) have emerged as transformative tools for addressing challenges in genomic research. By training ML models on large datasets, it is possible to detect patterns and anomalies with remarkable accuracy. Applying these techniques to the post-sequencing verification phase offers a promising approach to overcoming the limitations of traditional methods. The integration of AI with DNA sequencing workflows not only enhances reliability but also streamlines operations, paving the way for new discoveries in genomics.



\section{Objectives}
\label{sec:Objectives}

The primary goal of this thesis is to develop a tool that automates the verification of alignment results obtained from Sanger sequencing. The specific objectives include:

\begin{itemize}
  \item Designing an AI algorithm capable of detecting anomalies and homologous sequences in alignment data.
  \item Training the algorithm using real-world datasets from ABI3730xl sequencers 96-well plates.
  \item Designing an AI algorithm capable of detecting anomalies and homologous sequences in alignment data.
  \item Training the algorithm using real-world datasets from ABI3730xl sequencers 96-well plates.
  \item Validating the tools performance against traditional manual verification methods.
  \item Deploying the tool via an online platform to ensure accessibility for laboratories worldwide.
\end{itemize}

\section{Structure}
\label{sec:Structure}

This document is organized as follows:

\begin{itemize}
  \item Introduction: Provides the motivation, context, objectives, and structure of the thesis.
  \item State of the Art: 
  \item Related Work: Reviews existing literature and technologies in Sanger sequencing and AI applications in genomics.
  \item Methodology: Details the design and implementation of the proposed AI driven tool, including data collection, model training, and deployment.
  \item Planning: Presents the planning of the proposed work, composed by desired checkpoints and finish dates.
  \end{itemize}
% subsection suggestions_bugs_and_feature_requests (end)
