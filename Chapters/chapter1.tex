%!TEX root = ../template.tex
%%%%%%%%%%%%%%%%%%%%%%%%%%%%%%%%%%%%%%%%%%%%%%%%%%%%%%%%%%%%%%%%%%%
%% chapter1.tex
%% NOVA thesis document file
%%
%% Chapter with introduction
%%%%%%%%%%%%%%%%%%%%%%%%%%%%%%%%%%%%%%%%%%%%%%%%%%%%%%%%%%%%%%%%%%%

\typeout{NT FILE chapter1.tex}%

\chapter{Introduction}
\label{cha:Introduction}

\prependtographicspath{{Chapters/Figures/Covers/}}

In this initial chapter, the motivation for automating critical aspects of the Sanger sequencing pipeline is explored in detail. We discuss the operational challenges faced by laboratory personnel in manually verifying DNA sequencing results, the potential for artificial intelligence (AI) to alleviate these challenges, and the overall significance of introducing automated solutions in the context of genomic research.

The chapter also establishes the context in which this thesis was developed, explaining the collaboration between the company and the university to address specific problems in sequencing workflows. The importance of minimizing errors, optimizing throughput, and improving accuracy in genomic sequencing is emphasized, setting the stage for the problem addressed in this thesis.

Following the context, the defined objectives of the thesis are outlined, detailing the deliverables and results expected from this research. Finally, a brief overview of the thesis document's structure is provided to guide the reader through the subsequent chapters, ensuring a clear understanding of the overall thesis plan.

\section{Motivation}
\label{sec:Motivation}

Artificial intelligence (AI) represents one of the most transformative advancements in modern science, reshaping various disciplines and enabling breakthroughs that were once deemed impossible. Since the earliest days of humanity, the quest to understand the world around us has driven innovation. Today, this quest extends to replicating human cognition and decision-making processes, as AI evolves to emulate the brain—the core processor of human rationality.

\subsection{AI in Modern Science}
AI has demonstrated its profound potential in numerous fields, including healthcare, engineering, and biology. In healthcare, AI systems have enhanced diagnostics and treatment planning \cite{esteva2017dermatologist}. Autonomous systems have revolutionized transportation and logistics, while optimization algorithms have improved efficiency across industries \cite{russell2010artificial}. Among these domains, biology stands uniquely poised to benefit from AI’s capabilities, particularly in genomics.

Biology, as a data-rich science, faces the challenge of analyzing and interpreting vast datasets with precision. The automation, optimization, and decision-making facilitated by AI unlock opportunities to achieve milestones previously unattainable. For example, AI has accelerated drug discovery, improved our understanding of genetic diseases, and optimized agricultural practices \cite{senior2020protein, jumper2021highly}.

\subsection{AI in Genomics}
One specific sub-field of biology that benefits immensely from AI is genomics. DNA sequencing has revolutionized biological sciences, enabling breakthroughs in personalized medicine, agricultural biotechnology, and evolutionary studies. Among sequencing methods, Sanger sequencing remains the gold standard due to its precision, yet it presents several operational challenges in high-throughput workflows.

\section{Context}
\label{sec:Context}

Given that AI has non-stop increased relevance in the market, many companies seek to leverage it's capabilities for specific services, not only creating market share, but also providing competition, innovation, and bolstering the scienfitic advancement of the field.
This thesis comes in context with one ambitious company that seeks to innovate AI's use in genomics.

\subsection{Introduction to STAB VIDA}

STAB VIDA is a leading company in the field of genomic sequencing, specializing in high-quality services for DNA and RNA analysis. With a strong focus on precision and reliability, STAB VIDA offers state-of-the-art sequencing technologies, including Sanger sequencing, which remains one of the most trusted methods for accurate DNA analysis \cite{Sanger1981}. The company caters to a wide range of clients, from academic researchers to pharmaceutical and agricultural industries, providing critical data for projects related to personalized medicine, evolutionary biology, and biotechnological innovations \cite{Fujimura2015}.

\subsection{Challenges in the Current Workflow}

Despite the advanced technology used by STAB VIDA, the manual processes associated with DNA sequencing hold significant challenges. In particular, the verification of DNA sequencing results in 96-well plates is an error-prone and time-consuming task. Laboratory personnel manually check each well for abnormalities, such as abnormal signal scores, cross-well contamination, or unexpected DNA sequences that could indicate potential errors in the sequencing process \cite{Chakravarthy2020}.

These tasks, although essential for ensuring accurate and high-quality sequencing results, introduce bottlenecks into the workflow. With a growing demand for rapid sequencing services, manual checks not only delay the overall process but also increase the likelihood of human error \cite{Batista2021}. This, in turn, can affect the reliability of sequencing results delivered to clients, impacting both client satisfaction and the company's reputation \cite{Kumar2018}.

\subsection{The Need for Automation at STAB VIDA}

Given the increasing complexity and volume of sequencing tasks, STAB VIDA recognizes the urgent need to automate certain aspects of the workflow. Specifically, the company seeks to leverage artificial intelligence to enhance the detection of anomalies in sequencing data \cite{Jiang2021}. AI has the potential to significantly improve the speed and accuracy of this process by automatically flagging wells that exhibit unusual patterns.

By automating the flagging process, STAB VIDA aims to:

\begin{itemize} 
  \item \textbf{Reduce manual labor}: Laboratory personnel will be freed from repetitive tasks and be able to focus on higher-order tasks that require human expertise, such as resolving flagged anomalies \cite{Yang2020}. 
  \item \textbf{Increase accuracy and throughput}: AI-driven anomaly detection will reduce human error, improving the overall accuracy of sequencing results delivered to the client and enabling faster turnaround times \cite{Chen2019}. 
  \item \textbf{Improve scalability}: Automation will allow STAB VIDA to handle larger volumes of sequencing data, facilitating growth and meeting the increasing demand for sequencing services \cite{Rajendran2021}.
\end{itemize}

The collaboration between STAB VIDA and the university has thus emerged as a critical step in addressing these challenges. This thesis, focusing on the development and deployment of an AI-powered tool for sequencing anomaly detection, directly aligns with STAB VIDA’s vision for operational efficiency and enhanced service quality \cite{Sheng2018}.

\subsection{AI-Powered Solutions for Sequencing Workflows}

The AI solution proposed in this thesis is designed to integrate seamlessly into the current sequencing workflow at STAB VIDA. By analyzing sequencing data from the company's ABI3730xl sequencers, the AI tool will automatically flag wells that deviate from expected patterns based on specific criteria, such as signal quality or sequence consistency \cite{Gokcay2019}. These flagged wells will then be quarantined for human review, where laboratory personnel can investigate and resolve any issues that might arise.

Key benefits of this AI-driven solution include:

\begin{itemize} \item \textbf{Automated anomaly detection}: The AI tool will automatically identify and flag wells exhibiting abnormalities, such as contamination, poor signal quality, or unusual DNA sequences \cite{Wu2020}. \item \textbf{Detailed reporting}: For each flagged well, the system will provide detailed explanations and reasons for the anomaly, enabling more efficient review and resolution \cite{Zhang2021}. \item \textbf{Integration with existing workflows}: The tool will be designed to work within STAB VIDA’s current sequencing infrastructure, ensuring minimal disruption to ongoing operations and enabling smooth adoption by laboratory staff \cite{Chen2021}. \end{itemize}

This thesis will therefore contribute to STAB VIDA’s mission to improve sequencing workflows, reduce operational inefficiencies, and enhance the quality and reliability of its sequencing services.
Firstly, some samples need human intervention, namely those that require purification or exhibit poor concentration levels, resulting in the need for dilution or correction. This human interaction introduces the risk of error, potentially affecting the output sequence and, in the case of purification, even contaminating neighboring samples.

\section{Objectives}
\label{sec:Objectives}

The primary objective of this thesis is to develop an AI-powered tool that automates the verification of alignment results in Sanger sequencing. This tool will address inefficiencies in the manual processes currently employed and enhance the overall reliability of sequencing workflows. Specific objectives include:

\begin{itemize}
  \item \textbf{Data Exploration and Preprocessing}: Use data science techniques to explore, determine, and select the relevant data for model training, ensuring the data is pre-processed effectively.
  \item \textbf{Model Development}: Research and Implement a robust AI algorithm capable of detecting anomalies in sequencing data, including signal inconsistencies, contamination, and unexpected DNA sequences.
  \item \textbf{Model Training and Validation}: Train and validate the AI algorithm using real-world datasets obtained from ABI3730xl sequencers, ensuring its applicability in practical laboratory settings.
  \item \textbf{Performance Evaluation}: Compare the AI tool's accuracy and efficiency against traditional manual verification methods, demonstrating its effectiveness in a real-world context.
  \item \textbf{Deployment Strategy}: Develop an accessible online platform for deploying the AI tool, ensuring seamless integration into existing laboratory workflows and maximizing user adoption.
\end{itemize}

Beyond these technical objectives, the thesis seeks to highlight the transformative potential of AI in genomics, demonstrating how interdisciplinary approaches can address complex challenges at the intersection of biology and computer science.

\section{Structure}
\label{sec:Structure}

This document is organized into four main chapters, each focusing on a critical aspect of the thesis:

\begin{itemize}
\item \textbf{Introduction}: Provides the motivation, context, objectives, and structure of the thesis.
\item \textbf{State of the Art \& Related Work}: Reviews existing literature and technologies in Sanger sequencing and AI applications in genomics.
\item \textbf{Proposal}: Details the design and implementation of the proposed AI driven tool, including data collection, model training, and deployment.
\item \textbf{Planning}: Presents the planning of the proposed work, composed by desired checkpoints and finish dates.
\end{itemize}

Each chapter is designed to guide the reader through the research journey, from identifying the problem to proposing and validating a solution. This structured approach ensures clarity and coherence, making the thesis accessible to both technical and non-technical audiences.

% subsection suggestions_bugs_and_feature_requests (end)
