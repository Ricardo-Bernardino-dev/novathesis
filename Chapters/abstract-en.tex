%!TEX root = ../template.tex
%%%%%%%%%%%%%%%%%%%%%%%%%%%%%%%%%%%%%%%%%%%%%%%%%%%%%%%%%%%%%%%%%%%%
%% abstract-en.tex
%% NOVA thesis document file
%%
%% Abstract in English([^%]*)
%%%%%%%%%%%%%%%%%%%%%%%%%%%%%%%%%%%%%%%%%%%%%%%%%%%%%%%%%%%%%%%%%%%%

\typeout{NT FILE abstract-en.tex}%

Since the dawn of mankind, our pursuit to understand the world has driven innovation and shaped the course of scientific discovery. Among the most profound achievements of this quest is the field of genomics, which has revolutionized our understanding of life itself. By decoding the genetic blueprint of organisms, genomics has unlocked insights into health, disease, evolution, and biodiversity, paving the way for transformative advancements in medicine, agriculture, and biotechnology.

Sanger sequencing remains a cornerstone of DNA analysis, yet its alignment results often necessitate labor-intensive manual verification to identify anomalies and inaccuracies. This process is not only time-consuming but also susceptible to human error, presenting significant challenges for laboratories handling high-throughput sequencing data. Addressing these limitations is critical for improving the efficiency and reliability of genomic workflows.

This master's thesis introduces an AI-driven tool designed to automate the verification of DNA sequence alignment results generated by the ABI3730xl sequencer. Leveraging machine learning models trained on real-world data from 96-well plates, the tool identifies anomalies, detects sequence homologies, and evaluates the acceptability of results with high precision. To ensure accessibility, the tool is deployed on an online platform, enabling remote usage by laboratories worldwide.

The proposed solution significantly enhances laboratory workflows by streamlining the verification process and reducing reliance on manual oversight. This not only minimizes errors but also accelerates data processing, supporting advancements in precision genomics and biotechnology. Additionally, the tool’s capability to detect inaccurately labeled wells further highlights its practical utility in high-throughput DNA analysis.

By addressing critical limitations in current practices, this work demonstrates the transformative potential of integrating artificial intelligence into genomic research. The tool contributes to the field by enabling more efficient and accurate DNA sequencing, paving the way for future innovations in genomics and biotechnology.

\keywords{
Sanger Sequencing \and
Artificial Intelligence \and
DNA Verification \and
Machine Learning \and
Genomic Workflows \and
High-Throughput Sequencing \and
Automation \and
Precision Genomics
}