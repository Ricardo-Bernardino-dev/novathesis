%!TEX root = ../template.tex
%%%%%%%%%%%%%%%%%%%%%%%%%%%%%%%%%%%%%%%%%%%%%%%%%%%%%%%%%%%%%%%%%%%%
%% abstract-en.tex
%% NOVA thesis document file
%%
%% Abstract in English([^%]*)
%%%%%%%%%%%%%%%%%%%%%%%%%%%%%%%%%%%%%%%%%%%%%%%%%%%%%%%%%%%%%%%%%%%%

\typeout{NT FILE abstract-en.tex}%

Sanger sequencing, while a cornerstone of DNA analysis, frequently yields alignment results that require labor-intensive manual verification to identify anomalies and inaccuracies. This process is not only time-consuming but also susceptible to human error, posing significant challenges for laboratories managing high-throughput sequencing data. Addressing this issue is critical for improving efficiency and reliability in genomic workflows.

This thesis introduces an AI-driven tool designed to automate the verification of DNA sequence alignment results produced by the ABI3730xl sequencer. By utilizing machine learning models trained on real-world data from 96-well plates, the tool identifies anomalies, detects sequence homologies, and determines the acceptability of results with a high degree of precision. To facilitate broader adoption, the tool is accessible via an online platform, enabling remote usage by laboratories worldwide.

The proposed solution significantly enhances laboratory workflows by streamlining the verification process and reducing reliance on manual oversight. This not only minimizes errors but also accelerates data processing, paving the way for advancements in precision genomics and biotechnology. The tool’s ability to identify inaccurately labeled wells further underscores its practical value in supporting high-throughput DNA analysis.

By addressing key limitations in current practices, this work demonstrates the transformative potential of integrating artificial intelligence into genomic research, contributing to a future of more efficient and accurate DNA sequencing.

% Palavras-chave do resumo em Inglês
% \begin{keywords}
% Keyword 1, Keyword 2, Keyword 3, Keyword 4, Keyword 5, Keyword 6, Keyword 7, Keyword 8, Keyword 9
% \end{keywords}
\keywords{
  Sanger Sequencing \and
  Artificial Intelligence \and
  DNA verification \and
  Machine Learning \and
  Genomic Workflows
}
