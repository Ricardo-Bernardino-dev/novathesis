%!TEX root = ../template.tex
%%%%%%%%%%%%%%%%%%%%%%%%%%%%%%%%%%%%%%%%%%%%%%%%%%%%%%%%%%%%%%%%%%%%
%% abstract-pt.tex
%% NOVA thesis document file
%%
%% Abstract in Portuguese
%%%%%%%%%%%%%%%%%%%%%%%%%%%%%%%%%%%%%%%%%%%%%%%%%%%%%%%%%%%%%%%%%%%%

\typeout{NT FILE abstract-pt.tex}%

A sequenciação de Sanger é um método fundamental na análise de ADN, mas os seus resultados de alinhamento frequentemente requerem uma verificação manual intensiva para identificar anomalias e imprecisões. Este processo, além de ser moroso, está sujeito a erros humanos, representando desafios significativos para laboratórios que lidam com dados de sequenciação em larga escala. Superar estas limitações é essencial para melhorar a eficiência e fiabilidade dos fluxos de trabalho genómicos.

Esta dissertação de mestrado apresenta uma ferramenta baseada em inteligência artificial (IA) concebida para automatizar a verificação dos resultados de alinhamento de sequências de ADN gerados pelo sequenciador ABI3730xl. Através da aplicação de modelos de aprendizagem automática treinados com dados reais de placas de 96 poços, a ferramenta identifica anomalias, deteta homologias entre sequências e avalia a aceitabilidade dos resultados com elevada precisão.

A solução proposta melhora significativamente os fluxos de trabalho laboratoriais ao simplificar o processo de verificação e reduzir a dependência da supervisão manual. Isto não só minimiza erros, como também acelera o processamento de dados, apoiando avanços na genómica de precisão e biotecnologia. Além disso, a capacidade da ferramenta para detetar poços incorretamente rotulados demonstra a sua utilidade prática na análise de ADN em larga escala.

Ao enfrentar limitações críticas nas práticas atuais, este trabalho destaca o potencial transformador da integração da inteligência artificial na investigação genómica. A ferramenta contribui para o campo ao permitir uma sequenciação de ADN mais eficiente e precisa, abrindo caminho para futuras inovações na genómica e biotecnologia.

\keywords{
Sequenciação de Sanger \and
Inteligência Artificial \and
Verificação de ADN \and
Aprendizagem Automática \and
Fluxos de Trabalho Genómicos \and
Sequenciação em Larga Escala \and
Automação \and
Genómica de Precisão
}
