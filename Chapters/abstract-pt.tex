%!TEX root = ../template.tex
%%%%%%%%%%%%%%%%%%%%%%%%%%%%%%%%%%%%%%%%%%%%%%%%%%%%%%%%%%%%%%%%%%%%
%% abstract-pt.tex
%% NOVA thesis document file
%%
%% Abstract in Portuguese
%%%%%%%%%%%%%%%%%%%%%%%%%%%%%%%%%%%%%%%%%%%%%%%%%%%%%%%%%%%%%%%%%%%%

\typeout{NT FILE abstract-pt.tex}%

Desde os primórdios da humanidade, a nossa busca para compreender o mundo tem impulsionado a inovação e moldado o curso da descoberta científica. Entre os feitos mais profundos desta busca está o campo da genómica, que revolucionou a nossa compreensão da vida em si. Ao decifrar o código genético dos organismos, a genómica revelou insights sobre saúde, doença, evolução e biodiversidade, abrindo caminho para avanços transformadores na medicina, agricultura e biotecnologia.

O sequenciamento Sanger é um método fundamental na análise de DNA, mas os seus resultados de alinhamento exigem frequentemente uma verificação manual, um processo moroso para identificar anomalias e imprecisões. Este processo não só consome tempo, como também está sujeito a erros humanos, representando um desafio significativo para laboratórios que lidam com dados de sequenciação de alto débito. Superar estas limitações é essencial para melhorar a eficiência e a fiabilidade dos fluxos de trabalho genómicos.

Esta tese de mestrado apresenta uma ferramenta baseada em inteligência artificial (IA) concebida para automatizar a verificação dos resultados de alinhamento de sequências de DNA gerados pelo sequenciador ABI3730xl. Utilizando modelos de aprendizagem automática treinados com dados reais de placas de 96 poços, a ferramenta identifica anomalias, deteta homologias de sequências e avalia a aceitabilidade dos resultados com elevada precisão. Para garantir acessibilidade, a ferramenta foi implementada numa plataforma online, permitindo a sua utilização remota por laboratórios em todo o mundo.

A solução proposta melhora significativamente os fluxos de trabalho laboratoriais ao agilizar o processo de verificação e reduzir a dependência de supervisão manual. Isto não só minimiza erros, como também acelera o processamento de dados, contribuindo para avanços na genómica de precisão e na biotecnologia. Além disso, a capacidade da ferramenta para identificar poços incorretamente rotulados reforça a sua utilidade prática na análise de DNA de alto débito.

Ao abordar limitações críticas das práticas atuais, este trabalho demonstra o potencial transformador da integração da inteligência artificial na investigação genómica. A ferramenta contribui para o campo ao possibilitar uma sequenciação de DNA mais eficiente e precisa, abrindo caminho para futuras inovações na genómica e na biotecnologia.

\keywords{
Sequenciação Sanger \and
Inteligência Artificial \and
Verificação de DNA \and
Aprendizagem Automática \and
Fluxos de Trabalho Genómicos \and
Sequenciação de Alto Débito \and
Automação \and
Genómica de Precisão
}
