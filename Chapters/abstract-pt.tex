%!TEX root = ../template.tex
%%%%%%%%%%%%%%%%%%%%%%%%%%%%%%%%%%%%%%%%%%%%%%%%%%%%%%%%%%%%%%%%%%%%
%% abstract-pt.tex
%% NOVA thesis document file
%%
%% Abstract in Portuguese
%%%%%%%%%%%%%%%%%%%%%%%%%%%%%%%%%%%%%%%%%%%%%%%%%%%%%%%%%%%%%%%%%%%%

\typeout{NT FILE abstract-pt.tex}%

A sequenciação Sanger, apesar de ser uma pedra angular na análise de DNA, frequentemente produz resultados de alinhamento que requerem verificação manual intensiva para identificar anomalias e imprecisões. Este processo não é apenas moroso, mas também suscetível a erros humanos, apresentando desafios significativos para laboratórios que gerem dados de sequenciação de alto desempenho. Abordar este problema é crucial para melhorar a eficiência e a confiabilidade nos fluxos de trabalho genómicos.

Esta tese apresenta uma ferramenta baseada em inteligência artificial (IA) concebida para automatizar a verificação dos resultados de alinhamento de sequências de DNA produzidos pelo sequenciador ABI3730xl. Utilizando modelos de aprendizagem automática treinados com dados reais de placas de 96 poços, a ferramenta identifica anomalias, deteta homologias de sequências e determina a aceitabilidade dos resultados com um elevado grau de precisão. Para facilitar a adoção ampla, a ferramenta é acessível através de uma plataforma online, permitindo a utilização remota por laboratórios em todo o mundo.

A solução proposta melhora significativamente os fluxos de trabalho laboratoriais ao simplificar o processo de verificação e reduzir a dependência da supervisão manual. Isto não apenas minimiza erros, mas também acelera o processamento de dados, abrindo caminho para avanços na genómica de precisão e na biotecnologia. A capacidade da ferramenta de identificar poços rotulados de forma incorreta destaca ainda mais o seu valor prático no suporte à análise de DNA de alto desempenho.

Ao abordar limitações-chave nas práticas atuais, este trabalho demonstra o potencial transformador de integrar inteligência artificial à investigação genómica, contribuindo para um futuro de sequenciação de DNA mais eficiente e preciso.

% E agora vamos fazer um teste com uma quebra de linha no hífen a ver se a \LaTeX\ duplica o hífen na linha seguinte se usarmos \verb+"-+… em vez de \verb+-+.
%
% zzzz zzz zzzz zzz zzzz zzz zzzz zzz zzzz zzz zzzz zzz zzzz zzz zzzz zzz zzzz comentar"-lhe zzz zzzz zzz zzzz
%
% Sim!  Funciona! :)

% Palavras-chave do resumo em Português
% \begin{keywords}
% Palavra-chave 1, Palavra-chave 2, Palavra-chave 3, Palavra-chave 4
% \end{keywords}
\keywords{
  Sequenciação Sanger \and
  Inteligência Artificial \and
  Verificação de DNA \and
  Aprendizagem Automática \and
  Fluxos de trabalho genómicos
}
% to add an extra black line
